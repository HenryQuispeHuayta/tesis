\chapter*{Resumen}

La presente tesis desarrolla un intérprete de pseudocódigo en español enfocado en la programación estructurada. Este proyecto tiene como objetivo principal proporcionar una herramienta accesible para personas que desean aprender programación sin tener conocimientos previos de inglés, facilitando así el entendimiento de conceptos básicos de la algoritmia y la lógica de programación.

En el marco teórico, se exploran los fundamentos de la programación estructurada, así como las diferencias entre compiladores e intérpretes. También se abordan conceptos esenciales para el diseño de lenguajes de programación y las metodologías de trabajo aplicadas en el desarrollo del proyecto.

El marco aplicativo detalla la implementación del intérprete, incluyendo el análisis léxico, sintáctico y semántico del pseudocódigo. Se describe el proceso de evaluación y ejecución del código, así como las pruebas realizadas para asegurar la funcionalidad, fiabilidad, usabilidad, eficiencia, mantenibilidad, portabilidad, seguridad y compatibilidad del sistema.

Finalmente, se presentan las conclusiones y recomendaciones, subrayando la efectividad del intérprete como herramienta educativa y proponiendo mejoras para futuros desarrollos.

Con esta tesis, se espera contribuir al campo de la educación en informática, proporcionando una solución práctica para facilitar el aprendizaje de la programación en español, promoviendo así la inclusión y el desarrollo de habilidades tecnológicas en una población más amplia.

\textbf{Palabras clave:} Pseudocódigo, intérprete, programación estructurada, educación.
