\chapter*{Abstract}

This thesis develops an interpreter for pseudocode in Spanish, focused on structured programming. The main objective of this project is to provide an accessible tool for people who wish to learn programming without prior knowledge of English, thereby facilitating the understanding of basic concepts in algorithms and programming logic.

The theoretical framework explores the fundamentals of structured programming, as well as the differences between compilers and interpreters. It also addresses essential concepts for the design of programming languages and the methodologies applied in the project's development.

The application framework details the implementation of the interpreter, including lexical, syntactic, and semantic analysis of the pseudocode. It describes the process of code evaluation and execution, as well as the tests conducted to ensure the functionality, reliability, usability, efficiency, maintainability, portability, security, and compatibility of the system.

Finally, the conclusions and recommendations are presented, highlighting the effectiveness of the interpreter as an educational tool and proposing improvements for future developments.

This thesis aims to contribute to the field of computer science education by providing a practical solution to facilitate learning programming in Spanish, thereby promoting inclusion and the development of technological skills in a broader population.

\textbf{Keywords:} Pseudocode, interpreter, structured programming, education.
