\begin{spacing}{1.5}
  \begin{center}
    \Large \textbf{Resumen}
  \end{center}
  Esta tesis desarrolla un intérprete de pseudocódigo en español enfocado en la programación estructurada, con el objetivo de proporcionar una herramienta accesible para aquellos que desean aprender programación sin conocimientos previos de inglés. Al utilizar el español como lengua principal, se busca superar las barreras lingüísticas y hacer que el aprendizaje sea más inclusivo para las personas de habla hispana. Este enfoque tiene el potencial de simplificar significativamente la comprensión de los conceptos básicos de algoritmos y programación, promoviendo así una mayor participación y compromiso en el campo de la informática. \\
  El marco teórico de esta investigación explora los fundamentos de la programación estructurada, que se basa en estructuras de control como secuencia, selección e iteración para construir programas claros y eficientes. También se examinan las principales diferencias entre compiladores e intérpretes, destacando que un intérprete, al ejecutar el código línea por línea, proporciona retroalimentación inmediata. Este aspecto es particularmente valioso en un contexto educativo, ya que facilita el aprendizaje al permitir a los usuarios identificar y corregir errores en tiempo real. Además, se discuten conceptos esenciales para el diseño de lenguajes, como la gramática y la semántica, así como las metodologías aplicadas en el desarrollo del proyecto. \\  
  En el marco de aplicación, la tesis detalla la implementación del intérprete, incluyendo varias etapas cruciales. El análisis léxico se encarga de descomponer el texto en tokens; el análisis sintáctico asegura que estos tokens se ajusten a la gramática del lenguaje; y el análisis semántico verifica que las operaciones sean lógicamente válidas. También se describe el proceso de evaluación y ejecución del código, y se presentan pruebas para garantizar la funcionalidad, fiabilidad, usabilidad, eficiencia, mantenibilidad, portabilidad, seguridad y compatibilidad del sistema. \\  
  Finalmente, la tesis concluye con un análisis exhaustivo de los resultados obtenidos, destacando la efectividad del intérprete como herramienta educativa y su potencial para un uso extendido. Se proponen sugerencias para trabajos futuros. Este trabajo busca contribuir al campo de la educación informática, avanzando en la accesibilidad y las oportunidades de aprendizaje para los hispanohablantes. \\
  \textbf{Palabras clave:} Pseudocódigo, intérprete, programación estructurada, educación. \\
  \textbf{Metodología:} Kanban.
\end{spacing}
