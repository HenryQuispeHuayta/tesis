\begin{spacing}{1.5}
  \begin{center}
    \Large \textbf{Resumen}
  \end{center}
  La presente tesis desarrolla un intérprete de pseudocódigo en español, centrado en la programación estructurada, con el objetivo de proporcionar una herramienta accesible para aquellos que desean aprender programación sin la necesidad de conocimientos previos de inglés. Al utilizar el español como idioma principal, se busca superar las barreras lingüísticas y hacer el aprendizaje más inclusivo para hispanohablantes. Este enfoque tiene el potencial de simplificar la comprensión de conceptos básicos de algoritmia y lógica de programación, promoviendo así una mayor participación en el ámbito de la informática. \\
  El marco teórico de esta investigación explora los fundamentos de la programación estructurada, que se basa en estructuras de control como secuencia, selección e iteración para construir programas claros y eficientes. Además, se analizan las diferencias clave entre compiladores e intérpretes, destacando que un intérprete, al ejecutar el código línea por línea, proporciona retroalimentación inmediata. Este aspecto es particularmente valioso en un contexto educativo, ya que facilita el aprendizaje al permitir a los usuarios identificar y corregir errores en tiempo real. También se abordan conceptos esenciales para el diseño de lenguajes de programación, tales como la gramática y la semántica, y se revisan las metodologías de trabajo aplicadas en el desarrollo del proyecto. \\
  En el marco aplicativo, se detalla la implementación del intérprete, que incluye varias etapas cruciales. El análisis léxico se encarga de dividir el texto en tokens, el análisis sintáctico verifica que estos tokens cumplan con la gramática del lenguaje, y el análisis semántico asegura que las operaciones sean lógicamente válidas. Además, se describe el proceso de evaluación y ejecución del código, y se presentan las pruebas realizadas para asegurar la funcionalidad, fiabilidad, usabilidad, eficiencia, mantenibilidad, portabilidad, seguridad y compatibilidad del sistema. \\
  Finalmente, la tesis concluye con un análisis de los resultados obtenidos, destacando la efectividad del intérprete como herramienta educativa. Se proponen mejoras para futuros desarrollos y se espera que esta investigación contribuya significativamente al campo de la educación en informática. El proyecto tiene el potencial de fomentar la inclusión y el desarrollo de habilidades tecnológicas en una población más amplia, haciendo que el aprendizaje de la programación sea más accesible para los hispanohablantes. \\
  \textbf{Palabras clave:} Pseudocódigo, intérprete, programación estructurada, educación. \\
  \textbf{Metodología:} Kanban.
\end{spacing}
