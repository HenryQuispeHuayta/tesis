\begin{spacing}{1.5}
  \begin{center}
    \Large \textbf{Abstract}
  \end{center}
  This thesis develops a Spanish pseudocode interpreter focused on structured programming, aiming to provide an accessible tool for those who wish to learn programming without prior knowledge of English. By using Spanish as the primary language, the goal is to overcome language barriers and make learning more inclusive for Spanish speakers. This approach has the potential to simplify the understanding of basic algorithmic and programming logic concepts, thereby promoting greater participation in the field of computing. \\
  The theoretical framework of this research explores the fundamentals of structured programming, which is based on control structures such as sequence, selection, and iteration to build clear and efficient programs. Additionally, the key differences between compilers and interpreters are analyzed, highlighting that an interpreter, by executing code line by line, provides immediate feedback. This aspect is particularly valuable in an educational context, as it facilitates learning by allowing users to identify and correct errors in real time. Essential concepts for programming language design, such as grammar and semantics, are also addressed, and the work methodologies applied in the project development are reviewed. \\
  In the application framework, the implementation of the interpreter is detailed, including several crucial stages. Lexical analysis is responsible for breaking the text into tokens, syntactic analysis verifies that these tokens comply with the language's grammar, and semantic analysis ensures that operations are logically valid. Additionally, the process of code evaluation and execution is described, and tests are presented to ensure the system's functionality, reliability, usability, efficiency, maintainability, portability, security, and compatibility. \\
  Finally, the thesis concludes with an analysis of the results obtained, highlighting the effectiveness of the interpreter as an educational tool. Suggestions for future improvements are proposed, and it is expected that this research will make a significant contribution to the field of computer education. The project has the potential to foster inclusion and the development of technological skills among a broader population, making programming learning more accessible to Spanish speakers. \\
  \textbf{Keywords:} Pseudocode, interpreter, structured programming, education. \\
  \textbf{Methodology:} Kanban.
\end{spacing}