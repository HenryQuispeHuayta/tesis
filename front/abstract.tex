\begin{spacing}{1.3}
  \begin{center}
    \Large \textbf{Abstract}
  \end{center}
  This thesis develops a Spanish pseudocode interpreter focused on structured programming, aiming to provide an accessible tool for those who wish to learn programming without prior knowledge of English. By using Spanish as the primary language, the goal is to overcome linguistic barriers and make learning more inclusive for Spanish-speaking individuals. This approach has the potential to significantly simplify the understanding of basic algorithm and programming concepts, thereby promoting greater participation and engagement in the field of computing. \\
  The theoretical framework of this research explores the fundamentals of structured programming, which relies on control structures such as sequence, selection, and iteration to build clear and efficient programs. Key differences between compilers and interpreters are also examined, highlighting that an interpreter, by executing code line by line, provides immediate feedback. This aspect is particularly valuable in an educational context, as it facilitates learning by allowing users to identify and correct errors in real-time. Additionally, essential concepts for language design, such as grammar and semantics, are discussed, along with the methodologies applied in the project development. \\
  In the application framework, the thesis details the implementation of the interpreter, including several crucial stages. Lexical analysis is responsible for breaking down the text into tokens; syntactic analysis ensures that these tokens conform to the language's grammar; and semantic analysis ensures that operations are logically valid. The process of evaluating and executing code is also described, and tests are presented to ensure the system's functionality, reliability, usability, efficiency, maintainability, portability, security, and compatibility. \\
  Finally, the thesis concludes with a thorough analysis of the results obtained, highlighting the effectiveness of the interpreter as an educational tool and its potential for widespread use. Suggestions for future work are proposed. This work aims to contribute to the field of computer education, advancing accessibility and learning opportunities for Spanish speakers. \\
  \textbf{Keywords:} Pseudocode, interpreter, structured programming, education. \\
  \textbf{Methodology:} Kanban.
\end{spacing}
