\chapter{CONCLUSIONES Y RECOMENDACIONES}

\section{Conclusiones}
De los resultados obtenidos llegamos a evidenciar el cumplimiento de todos los objetivos trazados en el presente trabajo de investigación, los cuales se resumen a continuación:
\begin{itemize}
  \item Se desarrolló un prototipo de intérprete de pseudocódigo en español para programación estructurada. Este prototipo permite la ejecución de código escrito en español, facilitando el aprendizaje y la comprensión de conceptos de programación sin la necesidad de conocimientos en inglés. La herramienta ha demostrado ser eficiente y precisa en su propósito, cumpliendo con el objetivo general del proyecto.
  \item Se implementó un analizador léxico capaz de reconocer y clasificar los elementos léxicos del pseudocódigo en español. El analizador maneja correctamente palabras clave, operadores, identificadores y constantes, lo que garantiza una identificación precisa de los tokens necesarios para el procesamiento del pseudocódigo.
  \item Se realizó la implementación de un analizador sintáctico que verifica la estructura gramatical del pseudocódigo. Este analizador lleva a cabo un análisis sintáctico adecuado y ha demostrado su capacidad para manejar diversas estructuras gramaticales sin errores significativos, asegurando su robustez y fiabilidad.
  \item Se elaboró una documentación completa y clara del intérprete, disponible en un repositorio público. La documentación incluye guías de instalación, tutoriales paso a paso y descripciones detalladas de las funcionalidades del analizador léxico y sintáctico. La inclusión de ejemplos prácticos y referencias cruzadas al código fuente facilita la comprensión y el uso del proyecto por parte de nuevos usuarios.
  \item Se desarrolló una extensión para el editor Visual Studio Code (VSCode) que complementa la funcionalidad del intérprete. Esta extensión ofrece una interfaz amigable y herramientas adicionales como resaltado de sintaxis, autocompletado y snippets personalizados, mejorando la experiencia del usuario y facilitando la escritura, ejecución y depuración del pseudocódigo en español.
\end{itemize}

\section{Recomendaciones}
Según lo realizado durante la elaboración de este trabajo, se recomienda a futuros trabajos similares considerar las siguientes sugerencias:
\begin{itemize}
  \item Desarrollar métodos de ejecución e instalación que cubran una amplia gama de sistemas operativos. Esto garantizará que el intérprete sea accesible y funcional en diversas plataformas, aumentando su versatilidad y usabilidad para un público más amplio.
  \item Enfocar los esfuerzos iniciales en asegurar que el sistema funcione correctamente y cumpla con los requisitos esenciales. La optimización de tiempos de ejecución y rendimiento puede ser abordada posteriormente, una vez que se haya establecido una base sólida y funcional, ya que los usuarios valoran la fiabilidad antes de la eficiencia en el tiempo de ejecución.
  \item Concentrar los esfuerzos en implementar un paradigma de programación principal. Si el tiempo lo permite, se pueden añadir otros paradigmas adicionales, pero es crucial que esto no desvíe el enfoque principal del desarrollo. Esto permitirá una implementación más efectiva y una mayor calidad en el paradigma principal elegido.
\end{itemize}